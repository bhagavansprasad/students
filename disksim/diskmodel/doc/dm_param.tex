
\subsection{Model Configuration}
Diskmodel uses libparam to input the following blocks of parameter data:

\begin{verbatim}
dm_disk
dm_layout_g1
dm_layout_g1_zone
dm_mech_g1
dm_layout_g2
dm_layout_g2_zone
dm_layout_g4
\end{verbatim}

\subsubsection{dm\_disk}

The outer \texttt{dm\_disk} block contains the top-level parameters
which are used to fill in the \texttt{dm\_disk\_if} structure.  The
only valid value for ``Layout Model'' is a \texttt{dm\_layout\_g1}
block and for ``Mechanical Model,'' a \texttt{dm\_mech\_g1} block.\\

\noindent 
\begin{tabular}{|p{\lpmodwidth}|p{\lpnamewidth}|p{0.5in}|p{0.5in}|}
\cline{1-4}
\texttt{dm\_disk} & \texttt{Block count} & int & required \\ 
\cline{1-4}
\multicolumn{4}{|p{6in}|}{
This specifies the number of data blocks. This capacity is exported by the
disk (e.g.,~to a disk array controller). It is not used directly
during simulation, but is compared to a similar value computed from
other disk parameters. A warning is reported if the values differ.
}\\ 
\cline{1-4}
\multicolumn{4}{p{5in}}{}\\
\end{tabular}\\ 
\noindent 
\begin{tabular}{|p{\lpmodwidth}|p{\lpnamewidth}|p{0.5in}|p{0.5in}|}
\cline{1-4}
\texttt{dm\_disk} & \texttt{Number of data surfaces} & int & required \\ 
\cline{1-4}
\multicolumn{4}{|p{6in}|}{
This specifies the number of magnetic media surfaces (not platters!) on
which data are recorded. Dedicated servo surfaces should not be
counted for this parameter.
}\\ 
\cline{1-4}
\multicolumn{4}{p{5in}}{}\\
\end{tabular}\\ 
\noindent 
\begin{tabular}{|p{\lpmodwidth}|p{\lpnamewidth}|p{0.5in}|p{0.5in}|}
\cline{1-4}
\texttt{dm\_disk} & \texttt{Number of cylinders} & int & required \\ 
\cline{1-4}
\multicolumn{4}{|p{6in}|}{
This specifies the number of physical cylinders. All cylinders that impact
the logical to physical mappings should be included.
}\\ 
\cline{1-4}
\multicolumn{4}{p{5in}}{}\\
\end{tabular}\\ 
\noindent 
\begin{tabular}{|p{\lpmodwidth}|p{\lpnamewidth}|p{0.5in}|p{0.5in}|}
\cline{1-4}
\texttt{dm\_disk} & \texttt{Mechanical Model} & block & optional \\ 
\cline{1-4}
\multicolumn{4}{|p{6in}|}{
This block defines the disk's mechanical model. Currently,
the only available implementation is \texttt{dm\_mech\_g1}.
}\\ 
\cline{1-4}
\multicolumn{4}{p{5in}}{}\\
\end{tabular}\\ 
\noindent 
\begin{tabular}{|p{\lpmodwidth}|p{\lpnamewidth}|p{0.5in}|p{0.5in}|}
\cline{1-4}
\texttt{dm\_disk} & \texttt{Layout Model} & block & required \\ 
\cline{1-4}
\multicolumn{4}{|p{6in}|}{
This block defines the disk's layout model.
}\\ 
\cline{1-4}
\multicolumn{4}{p{5in}}{}\\
\end{tabular}\\ 


\subsubsection{G1 Layout}
% \subsubsection{dm\_layout\_g1}

The \texttt{dm\_layout\_g1} block provides parameters for a first
generation (g1) layout model.\\

\input{dm_layout_g1_param.tex}

The \texttt{Zones} parameter is a list of zone blocks each of which
contains the following fields:\\

\input{dm_layout_g1_zone_param.tex}

\subsubsection{G1 Mechanics}
%\subsubsection{dm\_mech\_g1}

The \texttt{dm\_mech\_g1} block provides parameters for a first
generation (g1) mechanical model.\\

\input{dm_mech_g1_param.tex}

% -----------------------------------------------
% G2 Layout stuff

\subsubsection{G2 Layout}
% \subsubsection{dm\_layout\_g2}

The \texttt{dm\_layout\_g2} block provides parameters for a second
generation (g2) layout model.\\

\input{dm_layout_g2_param.tex}

The \texttt{Zones} parameter is a list of zone blocks each of which
contains the following fields:\\

\input{dm_layout_g2_zone_param.tex}

% -----------------------------------------------
% G4 Layout stuff

\subsubsection{G4 Layout}
% \subsubsection{dm\_layout\_g4}

The \texttt{dm\_layout\_g4} block provides parameters for a fourth
generation (g4) layout model.\\

\input{dm_layout_g4_param.tex}

\subsection{Seek Equation Definitions}


\subsubsection{Lee's Seek Equation}
\label{seek.lee}

\begin{math}
seekTime(x) = \left\{ \begin{array} {r@{\quad:\quad}l}
              0 & if x = 0 \\
              a\sqrt{x-1} + b(x-1) + c & if x > 0
              \end{array} \right., \mbox{where} \\
\\
\indent
x \quad\mbox{is the seek distance in cylinders,} \\
\indent
a = (-10 minSeek + 15 avgSeek - 5 maxSeek) / (3 \sqrt{numCyl}), \\
\indent
b = (7 minSeek - 15 avgSeek + 8 maxSeek) / (3 numCyl), \mbox{and}\\
\indent
c = minSeek.
\end{math}\\


\subsubsection{The HPL Seek Equation}
\label{seek.hpl}

\begin{tabular}{cc}
Seek distance     & Seek time \\ \hline
1 cylinder        & $V_6$ \\
$<$$V_1$ cylinders  & $V_2$ + $V_3$ * $\sqrt{dist}$ \\
$>=$$V_1$ cylinders & $V_4$ + $V_5$ * dist \\
\end{tabular}
, where {\it dist} is the seek distance in cylinders.
\newline
If $V_6 == -1$, single-cylinder seeks are computed using the second equation.
$V_1$ is specified in cylinders, and $V_2$ through $V_6$ are specified in 
milliseconds.

$V_1$ must be a non-negative integer, $V_2 \ldots V_5$ must be
non-negative floats and $V_6$ must be either a non-negative float or
$-1$.\\


\noindent\textbf{Format of an extracted seek curve}
\label{seek.extract}

An extracted seek file contains a number of (seek-time,seek-distance)
data points.
The format of such a file is very simple: the first line is
\[\texttt{Seek distances measured: <n>}\]
where \texttt{<n>} is the number of seek distances provided in the
curve.  This line is followed by \texttt{<n>} lines of the form
\texttt{<distance>, <time>} where \texttt{<distance>} is the seek
distance measured in cylinders, and \texttt{<time>} is the amount of
time the seek took in milliseconds. e.g.

\begin{verbatim}
Seek distances measured: 4
1,  1.2
2,  1.5
5,  5
10, 9.2
\end{verbatim}




